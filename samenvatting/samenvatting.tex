\chapter*{Samenvatting}
\addcontentsline{toc}{chapter}{Samenvatting}
\label{samenvatting}
\fancyhead[LO]{\textsc{Samenvatting}}
\fancyhead[RE]{\textsc{Samenvatting}}
\fancyhead[RO]{\thepage}
\fancyhead[LE]{\thepage}
\begin{otherlanguage*}{dutch}

Dit proefschrift handelt over onderzoek aan chemische bindingen met behulp van de Valence Bond (VB) theorie. In het inleidende hoofdstuk wordt een ruw overzicht gegeven van het ontstaan van de scheikundige discipline door mensen als Boyle, Mendeljev en Lewis. Vanaf de jaren twintig van de vorige eeuw werd de quantumchemie ontwikkeld na de introductie van de Schr\"{o}dinger vergelijking. Met de door Heitler en London in 1927 ge\"{i}ntroduceerde quantumchemische VB theorie is het mogelijk om een mathematisch model voor chemische bindingen te maken, gebaseerd op de visie van Lewis. Volgens Lewis bestaat een chemische binding uit een gedeeld elektronen paar tussen twee atomen. Een Lewis structuur komt in hoofdlijn overeen met een VB structuur.

Om berekeningen aan grotere moleculen mogelijk te maken is TURTLE, het VB programma van de Theoretische Chemie Groep te Utrecht, reeds in de negentiger jaren uitgebreid met een storingsrekening model, de \textit{pert} optie. Door deze optie kan de tijd, die benodigd is om de orbitals te optimaliseren, drastisch worden verlaagd. In hoofdstuk 2 wordt een aantal uitbreidingen van deze optie behandeld. In de vernieuwde versie van de optie is het mogelijk om de optimalisatiemethode per excitatie, per orbital of per orbitalklasse aan te geven, waardoor de gebruiker zelf meer invloed heeft op het optimalisatieproces. Tevens kan nu gebruik gemaakt worden van een Fock matrix, zoals dat ook al in MCSCF, de orthogonale equivalent van VBSCF, gebeurt. Aan de hand van een aantal testberekeningen worden de versnellingen en de veranderingen in het optimalisatiepatroon vergeleken.

Het dissociatiegedrag van moleculen wordt be\"{i}nvloed door het omringende medium. In hoofdstuk 3 wordt deze invloed geanalyseerd aan de hand van de dissociatie van vier chloorverbindingen met vari\"{e}rende polariteit, te weten chloromethaan, \textit{tert}-butylchloride, chlorosilaan en trimethylsilylchloride. De M-Cl binding, waarin M=C of Si, wordt opgerekt en voor een aantal bindingslengten (of afstanden) wordt met behulp van VB de energie uitgerekend, zowel voor de totale golffunctie, alsook voor de losse VB structuren, die het covalente en ionische bindingskarakter representeren. Op lange afstand is de covalente structuur (twee radicaal fragmenten) de enige die nog bijdraagt aan de totale golffunctie. Voor \textit{tert}-butylchloride betekent dissociatie in radicalen:
\\
\\
C(CH$_3$)$_3$Cl  $\rightarrow$ C(CH$_3$)$_{3}^{\bullet}$ + Cl$^\bullet$
\\
\\
\noindent{}Wanneer het oplosmiddeleffect wordt aangeschakeld middels het Polarizeerbaar Continu\"um Model (in dit geval het effect van water), dissoci\"{e}ren \textit{tert}-butylchloride, chlorosilaan en trimethylsilylchloride naar ionen, dat wil zeggen dat op lange afstand alleen de ionische structuur (twee ionische fragmenten) bijdraagt tot de totale golffunctie. Voor \textit{tert}-butylchloride wordt de reactie:
\\
\\
C(CH$_3$)$_3$Cl  $\rightarrow$ C(CH$_3$)$_{3}^{+}$ + Cl$^{-}$
\\
\\
\noindent{}Van de vier moleculen ondervindt alleen chloromethaan dusdanig weinig invloed van het oplosmiddel, dat voor dit molecuul op lange afstand nog steeds de covalente structuur de belangrijkste is. 

Veelal worden in VB orbitals, waarvan men veronderstelt dat ze geen invloed hebben op het actieve centrum, niet geoptimaliseerd, maar bevroren. In eerdere VB berekeningen gingen de onderzoekers ervan uit dat de C-H orbitals, die twee bindingslengten van het actieve centrum verwijderd waren, bevroren konden worden. In het laatste gedeelte van hoofdstuk drie wordt aangetoond dat dat niet het geval is. Deze orbitals worden betrokken bij het actieve centrum door hyperconjugatie, waardoor met name het \textit{tert}-butyl kation gestabiliseerd wordt. Ook het trimethylsilyl kation (Si(CH$_3$)$_{3}^{+}$) wordt gestabiliseerd door dit effect, zij het in mindere mate. 

In organometaalchemie is de vorming een binding niet beperkt tot twee atomen, maar kan deze zich uitspreiden over verscheidene ligande atomen. Voor de binding van metaal atomen aan organische ringen (liganden) is men meer ge\"{i}nteresseerd in de binding van twee fragmenten, dan in twee afzonderlijke atomen. In hoofdstuk 4 worden de VB berekeningen aan de moleculen CpAlH$_2$, CpSiH en CpSiH$_3$ behandeld. Van deze moleculen is onderzocht wat de gunstigste positie van het metaalhydride fragment boven de cyclopentadienyl ring is en wat het overwegende bindingskarakter van de Cp-M binging is. 

De totale energie geeft aan dat de metaalhydride groepen in CpAlH$_2$ en CpSiH zijn gesitueerd boven een C-C binding ($\eta^2$) en dat de SiH$_3$ groep in CpSiH$_3$ boven een enkel C--atoom ($\sigma$) gepositioneerd is. Voor alle drie de moleculen correspondeert de $\eta^5$ positie (boven het midden van de ring) met een maximum. Hessian berekeningen aan deze geometrie\"{e}n geven aan dat dit hogere orde zadelpunten zijn. Eerste orde zadelpunten (overgangstoestanden) bevinden zich op de rand van de ring tussen twee minima in. Voor CpAlH$_2$ en CpSiH zijn de overgangstoestanden de $\eta^3$ geometrie\"{e}n tussen aangrenzende $\eta^2$ geometrie\"{e}n en voor CpSiH$_3$ zijn het de $\eta^2$ geometrie\"{e}n tussen aangrenzende $\sigma$ geometrie\"{e}n.

Het karakter van de binding tussen Cp en het SiH$_3$ fragment is hoofdzakelijk $\sigma$ en in beperkte mate ionisch. Dit geldt ook voor CpAlH$_2$ en CpSiH, alhoewel in deze moleculen het covalente karakter opgebouwd is uit een $\sigma$ en een  $\pi$ bijdrage.

De laatste drie hoofdstukken van dit proefschrift hebben een gemeenschappelijk thema: aromaticiteit. In hoofdstuk 5 wordt het anorganische benzeen borazine (B$_3$N$_3$H$_6$) op drie manieren vergeleken met het iso-elektronische benzeen (C$_6$H$_6$). Tevens worden de moleculen B$_2$N$_2$H$_4$ en B$_4$N$_4$H$_8$ vergeleken met hun koolstof analogen, respectievelijk C$_4$H$_4$ en C$_8$H$_8$.

In eerste instantie wordt met een eenvoudig H\"uckel model gekeken naar het verschil in het $\pi$ systeem van de azabora verbindingen ten opzichte van de koolstofverbindingen. In de H\"uckel berekeningen worden voor B en N de volgende Coulomb integralen gebruikt: $\alpha_B = \alpha - \eta\beta$ en $\alpha_N = \alpha + \eta\beta$, waarin $\eta$ een parameter is die kan vari\"eren tussen 0 en 1. Als de $\eta$ parameter gelijk  is aan  nul dan zijn de H\"uckel $\pi$ orbitals gelijk aan die voor homonucleaire ringen. Wanneer $\eta$ toeneemt beginnen de bindende orbitals zich te concentreren op de elektronegatieve stikstof atomen en de anti-bindende orbitals op de elektropositieve boor atomen: de functies worden ``gelokaliseerd".

Ten tweede zijn voor deze moleculen kringstroom-dichtheidsdiagrammen gemaakt. In deze maps duiden delokale diatropische stromen (tegen de wijzers van de klok in) in het $\pi$ systeem op aromatisch karakter (benzeen, $4n+2$). In borazine is de stroming in het $\pi$ systeem diatropisch, maar gelokaliseerd rond de stikstof atomen.

In het laatste gedeelte van hoofdstuk 5 komen VB berekeningen aan bod. Uit de VBCI berekeningen blijkt dat de structuren met vrije elektronenparen op de stikstof atomen verreweg het belangrijkst zijn. Uit de delokale VBSCF berekeningen blijkt dat de 6 $\pi$ elektronen 3 vrije elektronenparen vormen op de stikstof atomen en dat de resonantie energie verwaarloosbaar klein is. De VB berekeningen bevestigen hiermee dus de lokalisatie van de orbitals rond de stikstof atomen, die zowel het H\"uckel model alsook de current-density maps aangeven.

In hoofdstuk 6 wordt, eveneens met behulp van VB \textit{en} kringstroom berekeningen, onderzocht in hoeverre anorganische benzeen analogen, waaronder het eerder genoemde borazine, benzeenachtig karakter bezitten. De tien onderzochte moleculen zijn onderverdeeld in twee categori\"{e}n, te weten homonucleaire (X$_6$H$_6$, waarin X=C of Si en X$_6$, waarin X=N of P) en heteronucleaire zesringen (X$_3$Y$_3$H$_6$, waarin (X,Y)=(B,N), (B,P), (Al,N) of (Al,P) en B$_3$Y$_3$H$_3$ (Y=O of S).

De homonucleaire moleculen bezitten benzeenachtig karakter, zoals resonantie tussen twee Kekul\'{e}-achtige structuren en ge\"{i}nduceerde diatropische kringstromen. Bijna alle heteronucleaire moleculen vertonen lokalisatie van de vrije elektronenparen op de elektronegatieve atomen en bovendien dragen de Kekul\'{e}-achtige structuren niet bij. Van deze moleculen vertoont alleen B$_3$P$_3$H$_6$ benzeenachtig karakter met een significante bijdrage van de Kekul\'{e} structuren, een behoorlijke resonantie energie en een redelijke diatropische kringstroom in de vlakke geometrie. Echter, wanneer het molecuul beschouwd wordt in de optimale, niet vlakke chair conformatie is te zien dat de kringstromen beginnen te lokaliseren. 

In hoofdstuk 7 wordt geanalyseerd in hoeverre de bekende regel van H\"{u}ckel, die van toepassing is op enkelvoudige koolstofringen ook toegepast kan worden op meervoudige koolstofringen, zoals pentaleen, indaceen en homologen met meer zesringen. Deze moleculen hebben allemaal 4$n$ $\pi$ elektronen en zouden volgens de regel van H\"{u}ckel anti-aromatisch zijn.

Ook voor deze moleculen zijn kringstroomdichtheids- en VB berekeningen gedaan. De kringstroom diagrammen laten zien, dat, naarmate het aantal tussengevoegde zesringen groter wordt, de paratropische kringstromen veranderen in diatropische, hetgeen wijst op groter aromatisch karakter in de grotere systemen. Uit de VBCI berekeningen blijkt dat de winst in resonantie energie in de grotere systemen groter is dan de energie die het kost om een biradicaal te vormen.

\end{otherlanguage*}
