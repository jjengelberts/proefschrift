\chapter*{Curriculum Vitae}
\addcontentsline{toc}{chapter}{Curriculum Vitae}
\fancyhead[LO]{\textsc{Curriculum Vitae}}
\fancyhead[RE]{\textsc{Curriculum Vitae}}
\fancyhead[RO]{\thepage}
\fancyhead[LE]{\thepage}
\begin{otherlanguage*}{dutch}
\label{cv}

\lettrine{\initial{J}}{}eroen Johan Engelberts werd geboren op 7 december 1972 te Rotterdam. Van september 1985 tot juni 1991 volgde hij het atheneum aan de CSG Comenius te Capelle aan den IJssel en deed examen in de vakken Nederlands, Duits, Engels, Wiskunde, Natuurkunde, Scheikunde en Biologie. In september 1991 begon hij zijn studie Scheikunde aan de Universiteit te Utrecht. Zijn eerste bijvak (1994) handelde over de synthese en analyse van fluorescerend gemaakte collo\"idale silicabolletjes bij de vakgroep Fysische en Collo\"idchemie van Prof.~dr.~H.~N.~W.~Lekkerkerker (zie List of Publications). In het voorjaar van 1995 deed hij zijn tweede bijvak. Tijdens dit bijvak programmeerde hij aan een systeem dat de chemische benaming van koolwaterstoffen weergaf voor molecuulgeometrie\"en uit een electronische database (Chemische Informatica, Dr.~A.~M.~F.~Hezemans). Voor zijn hoofdvak (1995-1997) gebruikte hij verschillende Machine Learning technieken (o.a. Artificial Neural Networks) voor de voorspelling van chemische eigenschappen van alcoholen uit Infraroodspectra (Chemometrie, Analytische Molecuul Spectrometrie, Prof.~dr.~J.~H.~van~der~Maas).

Van 1996 tot 1997 werkte hij voor Intis als Junior Project Manager Sales. In deze rol controleerde hij het correct functioneren van EDI (Electronic Data Interchange) vertaalscripts en assisteerde hij de helpdesk bij het onderzoeken en oplossen van vertaalfouten.  Van 1997 tot 2000 werkte hij voor Harbinger in verschillende functies. Van 1997 tot 1998 werkte hij op hun helpdesk als Customer Support Representative, van 1998 tot 1999 als Network Operator en van 1999 tot 2000 als Sales Engineer. In de periode 2000-2001 was hij werkzaam bij Lift-Off als Sales Engineer.

In het najaar van 2000 keerde hij terug naar de Universiteit te Utrecht om zich toe te leggen op zijn promotieonderzoek. De resultaten worden beschreven in dit proefschrift, dat tot stand kwam onder dagelijkse begeleiding van Dr.~J.~H.~van~Lenthe en in nauwe samenwerking met Prof.~dr.~L.~W.~Jenneskens en Dr.~R.~W.~A.~Havenith.

Sinds mei 2005 is hij werkzaam als adviseur bij SURFsara te Amsterdam.
\end{otherlanguage*}
